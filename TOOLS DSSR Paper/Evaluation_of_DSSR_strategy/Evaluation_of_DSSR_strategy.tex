%\pagenumbering{arabic}
%\setcounter{page}{1}

\section{Evaluation of DSSR strategy}
To evaluate the new DSSR strategy in terms of performance we performed extensive experiments. To get a clear view of its performance we determined the comparative performance of DSSR strategy with pure random and random plus strategy by applying them to similar systems under identical conditions. Performance was measured in terms of: the ability of a strategy to find maximum number of faults in a fixed number of tests and the time taken by each strategy to execute 10,000 tests for Group A and Group B respectively. In our experiments we gave due weightage to the number of faults as well as the time of execution because a strategy might be good in finding higher number of faults but may require more time to find these faults, thus not considered satisfactory in the field where emphasis is on speedy and accurate results. The number of tests were kept constant at 10,000 to get a fair competition among the two strategies otherwise both the strategies were capable of finding all the faults in the given SUT if the number of tests were increased to a reasonably higher level. Additional time taken to execute the test by each strategy was given due consideration in order to determine the number of test cases in a given time irrespective of the number of faults. 