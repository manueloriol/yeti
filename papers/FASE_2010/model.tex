\section{Model}\label{sec:model}

YETI uses four main notions (see Figure~\ref{fig:def}): \textit{routines}, \textit{types}, 
\textit{variables}, and \textit{modules}. A type is a collection 
of variables and routines that return values�(of that type). A 
routine is a computation unit that uses variables of a given type and 
returns values of a given type. A variable is a couple between a 
label and a value. A module is a collection of routines. 

\begin{figure}[h]
\[
\begin{array}{l c l r}
\multicolumn{3}{l}{v,v_{0},...,v_{k}}&~~values\\
\multicolumn{3}{l}{n,n_{0},...,n_{k},r}&~~names\\
\multicolumn{3}{l}{f,f_{0},...,f_{k},r}&~~failures\\
R&::=&(v_{1}:T_{1} \times...\times v_{k}:T_{k}) \rightarrow v_{0}:T_{0}&~~routines\\
V&::=&(n,v)&~~variables\\
M&::=&(R_{1},...,R_{k})&~~modules\\
T&::=&((R_{1},...,R_{k}),(V_{1},...,V_{l}))&~~types\\
data &::=&R|V|M|T&~~data\\
L&::=&()|(data_{1},...,data_{n})|L_{1} \oplus L_{2}|L \oplus data&~~lists\\
\end{array}
\]
\caption{YETI meta-model definitions.}\label{fig:def}
\end{figure}

Note that routines contain typing indications. These are the information needed 
by the infrastructure to make calls and query it to create or store instances. 

Subtyping ($\subtype$) is a relationship between two types and implies
that the two types conform as shown in the lower part of Figure~\ref{fig:subtype} (PROP).

\begin{figure}[h]
\begin{center}
\[
\begin{array}{c}
T_{1}\subtype T_{2}~~~T_{2}\subtype T_{3}\\
\hline
{T_{1}\subtype T_{3}} \\
\end{array}~~(TRANS)
\]\[
T\subtype T~~(REFL)
\]

\[
\begin{array}{c}
T_{1}\subtype T_{2}~~~T_{1}=((R_{11},...,R_{1k}),(V_{11},...,V_{1l}))~~~T_{2}=((R_{21},...,R_{2m}),(V_{21},...,V_{2n}))\\ 
\hline
{\forall i \exists j | R_{1i}\in R_{2j}~~~\forall i \exists j | V_{1i}\in V_{2j}}\\
\end{array}~~(PROP)
\]
\end{center}
\caption{Subtyping.}\label{fig:subtype}
\end{figure}

Note that the rule $(PROP)$ is non-standard for the reason that our model is descriptive 
of the actual programs and valid type indications rather than actual types. As an example,
a variable could be classified as an instance of a super type of its defining type and not 
as an instance of its defining type. 
If this model defined a programming 
language it would be incomplete. In our case, it is however irrelevant as we do not intend 
to be complete: we only aim at soundness -- to make calls that would not fail because of typing 
issues.

\subsection{Modelling Random Testing}
To model random testing we add primitives to the simple model outlined above.
Figure \ref{fig:inst} presents the primitives needed by random testing.
Unless specified otherwise, each instruction has a usual definition.

\begin{figure}[h]
\[
\begin{array}{r c l}
primitive &::=& selectVariableFromType(T)\\
&|& addVariableToType(T,V)\\
&|& selectRoutineFromModule(M)\\
&|& executeRoutineWithArguments(R,l)\\
&|& oracle(primitive)\\
&|& report(f)\\
instruction&::=& primitive\\
&|& n \textbf{:=} primitive | n \textbf{:=} data |n \textbf{:=} L\\
&|& instruction~\textbf{;}~instruction\\
&|&\textbf{forall}~n~\textbf{in}~list~\textbf{do}~instruction~\textbf{end}\\
&|&\textbf{ifIsFailure}~instruction~\textbf{then}~instruction~\textbf{else}~instruction\\
&|&\textbf{whileNotFinished} \{ instruction \}
\end{array}
\]
\[
\begin{array}{r l}
selectVariableFromType:& T \mapsto V\\
addVariableToType:& T\times V\mapsto T\\
selectRoutineFromModule:& M\mapsto R\\
executeRoutineWithArguments:& R\times l \mapsto V|f\\
oracle:& V | f \mapsto V | f\\
report:& f \mapsto \emptyset\\
\end{array}
\]
\caption{YETI primitives and instructions.}\label{fig:inst}
\end{figure}

Figure~\ref{fig:randtest} shows how YETI tests code at random using the model defined in Figure~\ref{fig:inst}. 

\begin{figure}[h]
\[
\begin{array}{l}
whileNotFinished\{\\
~~r\textbf{:=}selectRoutineFromModule(M_{underTest});\\
~~l\textbf{:=}();\\
~~\textbf{forall}~t~\textbf{in}~(T_{1}... T_{n})|i\neq 0~and~r==(v_{1}:T_{1} \times...\times v_{k}:T_{k})\rightarrow v_{0}:T_{0}\\
~~\textbf{do}~\\
~~~l\textbf{:=}l\oplus selectVariableFromType(t)~\\
~~\textbf{end};\\
~~res\textbf{:=}oracle(executeRoutineWithArguments(r,l));\\
~~\textbf{ifIsFailure}~res~\textbf{then}~report(res)~\textbf{else}~addVariableToType(T_{0})\\
\}\\
\end{array}
\]
\caption{Main random testing algorithm.}\label{fig:randtest}
\end{figure}

While instructions have a precise semantics we assume that primitives are actually user-defined.
This allows the easy definition of user-specific strategies and bindings.
For example, in the case of a pure random strategy on Java, the primitives would 
have the following semantics: 
\begin{itemize}
\item $selectVariableFromType(T)$ selects one of the variables of $T$ at random and returns it.
\item $addVariableToType(T,V)$ adds $V$ to $T$ and returns the result.
\item $selectRoutineFromModule(M)$ returns one of the routines in $M$ at random.
\item $executeRoutineWithArguments(R,l)$ makes the actual execution of the code and returns a created variable (if any).
\item $oracle(V|f)$ decides on the validity of a failure or a value according to the semantics of the target programming language.
\item $report(f)$ aggregates failures and only actually report unique failures.
\end{itemize}

Other semantics are however possible. For example Random+~\cite{} either picks selects values in the existing pool of object, generate new ones on the fly, or take interesting values. This would then mean a different semantics for $selectVariableFromType$.

