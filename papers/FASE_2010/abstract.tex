Random testing is a simple, effective technique for finding faults. It is usually considered as a purely automatic technique that does not need any input from the tester and only reports results once the testing session is over.

This article presents the York Extensible Testing Infrastructure (YETI), a random testing tool implemented in Java. YETI provides a strong decoupling between the strategies and the actual code, making its engine language agnostic. The tool runs at a high level of performances with $10^6$ calls per minute on Java code. It benefits from a graphical user interface and allows testers to orient the testing process in real-time during the testing session. This is useful to abort or modify the parameters of the testing session when needed. We illustrate the efficiency of YETI  with a study testing all classes in java.lang and some classes in the well known open source project iText.
 