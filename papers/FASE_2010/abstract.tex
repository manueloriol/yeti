Random testing is traditionally viewed as an inefficient technique. In the past few years, however, it was demonstrated that it is quite effective at finding faults. 

This article presents the York Extendible Testing Infrastructure (YETI), a random testing tool implemented in Java that allows to test code written in multiple programming languages (currently Java, JML and .NET). YETI provides a strong decoupling between the strategies and the actual language binding. The tool exhibits unparalleled performances with around $10^6$ calls per minute on Java code. It also benefits from a graphical user interface that allows test engineer to orient the testing process while testing to discover more faults. We illustrate the efficiency of such a tool  with a study testing all testable classes in java.lang and reports on faults found.
