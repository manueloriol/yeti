Random testing is a simple technique quite effective at finding faults.
There is to date no single system that would cope with several programming languages and provide some level of reuse of code and concepts. There is also very little interaction between test engineers and the random testing platform while testing.

This article presents the York Extendible Testing Infrastructure (YETI), a random testing tool implemented in Java that allows the testing of code written in multiple programming languages (currently Java, JML and .NET). YETI provides a strong decoupling between the strategies and the actual language binding. The tool exhibits unparalleled performances with around $10^6$ calls per minute on Java code. It also benefits from a graphical user interface that allows test engineer to orient the testing process while testing. We illustrate the efficiency of such a tool  with a study testing all classes in java.lang and some classes in a well known open source project.
