\section{Introduction}\label{sec:intro}

Automated random testing is a methodology often neglected by 
programmers and software testers because it is deemed as overly
simple. It has however advantages over other techniques because 
it is completely unbiased and allows the execution of a high number 
of calls over a short period of time.


The York Extendible Testing Infrastructure (YETI) provides a framework 
for executing random testing sessions. The main characteristics of 
YETI is that it supports multiple programming languages through a 
language-agnostic meta model. Various testing strategies apply
to all supported languages thanks to a strong decoupling between the 
strategies used and the programming language binding. 

Oracles are language-dependent. In the presence of specifications YETI checks 
inconsistencies between the code and the specifications. In the case of 
languages such as plain Java, if programmers use \texttt{assert} statements 
violations are interpreted as failures. If programmers do not use assert 
statements a testing session with YETI is then  robustness test that
consider undeclared runtime exceptions thrown as failures.

Unlike competitors, YETI also support a graphical user interface that 
allows test engineers to monitor the testing session and modify some 
of its characteristics while testing. To validate our approach we made 
one million calls at random on each considered class and tested 
all classes in \texttt{java.lang} as well as in ...


Section~\ref{sec:model} presents YETI's meta-model and its main algorithms.
Section~\ref{sec:implem} describes its current implementation.
Section~\ref{sec:evaluation} evaluates YETI.
Section~\ref{sec:rw} presents related work.
We eventually conclude in Section~\ref{sec:conc}.