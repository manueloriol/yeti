\section{Introduction}
The York Extensible Testing Infrastructure (YETI) is a language agnostic 
random testing tool. Its reference implementation tests Java programs
and is able to perform over one million method calls per minute on fast library
code. YETI found bugs in the \texttt{java.lang}, iText\footnote{http://itextpdf.com/} and many more 
programs. YETI, however, suffers from two issues:
\begin{enumerate}
\item \textbf{Performances.} On slow code, the performances can drop as low as $10^{3}$
method calls per minute.\footnote{Observed while testing the iText library in 
September 2009} While this might seem to be reasonable performances, experience shows 
that a plateau in the number of faults found for a given class is generally only reached after 
$10^{5}$ method calls. For large projects (100+ classes) this might mean that testing 
code on a single computer might not reach a stable point overnight. It also has the advantage of
using multiple seeds for the pseudo-random number generator, which previously showed better
performances than using only one seed over long running sessions~\cite{CPLOM:08:PRTOOS}.
\item \textbf{Security.} The Java binding of YETI relies on the Java security model to forbid 
undesirable side-effects. This is a reasonable assumption in most cases. It is however
sometimes undesirable because some programs need to access files, sockets, or drivers 
directly. It is also not possible to make such assumptions in case YETI is testing 
programs written in C.
\end{enumerate}

Distributing YETI over the cloud solves these two issues: (1) testing classes in parallel
allows to distribute the testing of 100 classes on 100 cores, resulting in a 10mn test 
rather than a 10000-minute one; (2) testing classes in a dedicated virtual machine, devoid of sensitive 
information, means that security restrictions can be relaxed.

This article presents a prototype of the cloud implementation of YETI. The prototype relies
on the map/reduce implementation of Hadoop to distribute testing sessions on remote machines 
and recombine them at the end of a desired testing time.

Section~\ref{sec:architecture} present the cloud implementation of YETI. Section~\ref{sec:future} 
presents the future developments of YETI on the cloud. Section~\ref{sec:rw} describe related work.
We eventually conclude in Section~\ref{sec:conc}.