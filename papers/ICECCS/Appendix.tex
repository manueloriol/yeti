\section*{Appendix}
\begin{verbatim}
using System;
using System.Collections.Generic;
using System.Diagnostics.Contracts;

namespace YetiTestAssembly1
{
    class Object1
    {
        public double doubleAttr1;
        public String s1;

        public Object1()
        {
            doubleAttr1 = 10.00;
            s1 = "";
        }

        public int methodReturnInt1(int r, int t)
        {
            Contract.Ensures(Contract.Result<int>() != 0);
            r = t;
            return r;
        }

        public void setDoubleAttr1(double d)
        {
            Contract.Ensures(d > 10.5);
            doubleAttr1 = d;
        }

        public String returnStringValue1()
        {
            return "A test String Object1";
        }

        public float methodForFloat1(Object1 ob, char c, long l)
        {
            Contract.Requires(ob != null && c != 'm');
            Contract.Ensures(l > 100);
            float f = (Single)l;
            return f;
        }

        public static void addingTwoInts1(int a, int b)
        {
            a = a + b;
        }

        public char returnCharValue1()
        {
            return 'c';
        }

        static void Main(string[] args)
        {
            Console.WriteLine("Starting Object1 Main method");
        }
    }
}
using System;
using System.Collections.Generic;
using System.Diagnostics.Contracts;

namespace YetiTestAssembly2
{
    class Object2
    {
        public double doubleAttr2;
        public Object2 ob2;

        public Object2()
        {
            doubleAttr2 = 55.22;
            ob2 = null;
        }

        public int methodReturnInt2(int r, int t)
        {
            Contract.Ensures(Contract.Result<int>() > 5);
            r = t;
            return r;
        }

        public void setOb2Attr(Object2 d)
        {
            Contract.Ensures(d !=null);
            ob2 = d;
        }

        public String returnStringValue2()
        {
            return "A test String Object2 fffff///***";
        }

        public float methodForFloat2(double ob, int l)
        {
            Contract.Requires(ob != 0.0);
            Contract.Ensures(Contract.Result<float>() != 0);
            float f = (Single)(l*ob);
            return f;
        }

        public static void addingTwoShorts2(short a, short b)
        {
            int h = a + b;           
        }

        public char returnCharValue2()
        {
            return 'U';
        }

        public Object2 genOb2()
        {
            return new Object2();
        }

        static void Main(string[] args)
        {
            Console.WriteLine("Starting Object2 Main method");
        }
    }
}
using System;
using System.Collections.Generic;
using System.Diagnostics.Contracts;
namespace YetiTestAssembly3
{
    class Object3
    {
        public double doubleAttr3;
        public int intAttr3;
        public Object3 ob3;

        public Object3()
        {
            doubleAttr3 = 333.333;
            intAttr3 = 1;
            ob3 = null;
        }

        public Object3(int t, double d, Object3 o)
        {
            doubleAttr3 = d;
            intAttr3 = t;
            ob3 = o;
        }

        public static int methodReturnInt3(int r, int t)
        {
            Contract.Requires(t != 4);
            Contract.Ensures(Contract.Result<int>() > 1);
            r = t;
            return r;
        }

        public void setOb3Attr(Object3 d, char c, String s)
        {
            Contract.Ensures(d !=null && c!='m');
            ob3 = d;
        }

        public String returnStringValue3()
        {
            return "ggg333nnn";
        }

        public float methodForFloat3(double ob, int l)
        {
            Contract.Requires(ob != 0.0);
            Contract.Ensures(Contract.Result<float>() != 100);
            float f = (Single)(l*ob);
            return f;
        }

        public static void dividingTwoShorts3(short a, short b)
        {
            Contract.Assert(b!=0);
            short h;
            h = Convert.ToInt16(a/b);
        }

        public char returnCharValue3()
        {
            return 'Y';
        }

        public Object3 genOb3()
        {
            return new Object3();
        }

        static void Main(string[] args)
        {
            Console.WriteLine("Starting Object3 Main method");
        }
    }
}
\end{verbatim}