\section{Introduction}\label{sec:intro}

Automated random testing is a methodology often neglected by 
programmers and software testers because it is deemed as overly
simple. It has, however, advantages over other techniques because 
it is completely unbiased and allows the execution of a high number 
of calls over a short period of time.


The York Extendible Testing Infrastructure (YETI) provides a framework 
for executing random testing sessions. The main characteristics of 
YETI is that it supports multiple programming languages through a 
language-agnostic meta model. Various testing strategies apply
to all supported languages thanks to a strong decoupling between the 
strategies used and the programming language binding. 

Oracles are language-dependent. In the presence of specifications YETI checks 
inconsistencies between the code and the specifications. In the case of 
languages such as .NET, code-contracts\footnote{http://research.microsoft.com/en-us/projects/contracts/} can be used to direct the testing 
of the code. In case a precondition of the method under test is violated, then we 
do not interpret it as a failure. In case no precondition is violated, any 
exception can be interpreted as a bug if it is not declared in the documentation. 
If programmers do not use contracts, a testing session of .NET programs with YETI 
is then a robustness test that reports all runtime exceptions. Because .NET does 
not declare runtime exceptions, all triggered must then be compared with exceptions 
declared in the documentation 
 

Unlike competitors, YETI also support a graphical user interface that 
allows test engineers to monitor the testing session and modify some 
of its characteristics while testing. To validate our approach we classes using 
code-contracts and tried to find seeded bugs in those classes. We also applied 
our technique to the .NET system libraries. 


Section~\ref{sec:model} presents informally YETI's meta-model and its main algorithms.
Section~\ref{sec:implem} describes its current implementation for .NET.
Section~\ref{sec:evaluation} evaluates the YETI binding for .NET.
Section~\ref{sec:rw} presents related work.
Section~\ref{sec:fut} describes future work.
We eventually conclude in Section~\ref{sec:conc}.
