In the recent years, random testing has drawn interest from the community because it is an effective and predictable way of finding faults. Two main issues persist: (1) to overcome its random nature a random testing tool must perform large numbers of tests, (2) there is currently no accepted stopping criterion for random testing tools.

This article presents the York Extensible Testing Infrastructure (YETI), a random testing tool whose reference implementation is in Java. The tool makes $10^6$ calls per minute on Java code, the highest performance to date for such a tool. It also benefits from a graphical user interface which allows testers to interact with the testing process in real-time, potentially deciding to stop a testing session if it seems that finding new bugs is unlikely. We illustrate how YETI works with a study testing all classes in java.lang and some classes in the well known open source project iText.
 